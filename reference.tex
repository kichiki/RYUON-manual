% RYUON Reference Manual
% Copyright (C) 2007-2008 Kengo Ichiki <kichiki@users.sourceforge.net>
% $Id: reference.tex,v 1.1 2008/07/11 04:18:35 kichiki Exp $

%\documentclass{book}
\documentclass[twocolumn]{book}

\usepackage[dvips]{graphicx}
\usepackage{amsmath}
% some trick to prevent "Too many math alphabets" error for "bm" package
\newcommand\bmmax{0}
\usepackage{bm}
\usepackage{times}
\usepackage{txfonts}
\usepackage{mathrsfs}

\begin{document}

\title{RYUON Reference Manual}
\author{Kengo Ichiki
 \thanks{e-mail: kichiki@users.sourceforge.net}}
\date{\today}

\maketitle

\tableofcontents

\chapter{Introduction}
This document describes details of some numerical implementations 
of {\bf RYUON} (aiming for developers).


There are other documents:
\begin{itemize}
\item ``RYUON User's Manual'':
  describing overview and some tutorials of {\bf RYUON} (for users).
\item ``RYUON Reference Manual'':
  describing numerical implementations 
  of {\bf RYUON} (for developers).
\item ``Theory of RYUON'':
  a textbook describing theoretical background of {\bf RYUON}
  (for researchers).
\end{itemize}


\chapter{General Mobility Problem -- ${\bf A}\cdot\bm{x}$ Calculation}
\label{chp:atimes}


\chapter{Mobility Problem}
\label{chp:mob}


\chapter{Resistance Problem}
\label{chp:res}


\chapter{Mixed Problem}
\label{chp:mix}


\chapter{Lubrication}
\label{chp:lub}

\section{Top Layer}
The lubrication forces for $N$ (all) particles 
The lubrication forces for $N$ (all) particles 
\begin{equation}
  \bm{F}
  =
  \bm{L}
  \cdot
  \bm{U}
  ,
\end{equation}
is calculated by
\begin{equation}
  {\tt calc\_lub\_\langle d\rangle\langle ver\rangle\ (sys, u, f)}
  ,
\end{equation}
which is called by the top-level function
\begin{equation}
  {\tt solve\_\langle type\rangle\_lub\_\langle d\rangle\langle ver\rangle\ ()}
  ,
\end{equation}
for mobility, resistance, and mixed problems ({\tt type}) 
in 2- and 3-dimensions ({\tt d}) 
with F, FT, and FTS versions ({\tt ver}).

The function ${\tt calc\_lub\_\langle d\rangle\langle ver\rangle\ ()}$ 
has particle-loops.
Under the periodic boundary condition, particles in the image cells 
are also taken into account.

In the loop for particle-particle pairs, 
either 
\begin{equation}
  {\tt calc\_lub\_\langle ver\rangle\_2b()}
\end{equation}
for monodisperse system or
\begin{equation}
  {\tt calc\_lub\_\langle ver\rangle\_2b\_poly()}
\end{equation}
for polydisperse system is called.


\section{Polydisperse Systems}
All information necessary to calculate the interactions for 
polydisperse systems are stored in the struct {\tt twobody\_f} 
for each pair characterized by the size ratio 
$\lambda := a_2 / a_1$. 

\begin{itemize}
\item {\tt calc\_lub\_[f,ft,fts]\_2b\_poly()} in f.c, ft.c, fts.c:\\
  calculate lubrication forces (and torques and stresslets) by 
  \begin{equation}
    \left[
      \begin{array}{c}
        \bm{F}^{(1)} \\ \bm{F}^{(2)}
      \end{array}
    \right]
    =
    \bm{L}_{\text{2B}}
    \cdot
    \left[
      \begin{array}{c}
        \bm{U}^{(1)} \\ \bm{U}^{(2)}
      \end{array}
    \right]
    ,
  \end{equation}
  for particles {\tt i1} and {\tt i2}
  at {\tt x1[3]} and  {\tt x2[3]}, respectively.
  This calculation is done in the SD scaling, 
  achieved by the function 
  {\tt scalars\_res\_poly\_scale\_SD()}.
  The function 
  {\tt scalars\_lub\_poly\_full()} is called 
  to obtain the scalar functions of the lubrication matrix.
\item {\tt scalars\_lub\_poly\_full()} in minv-poly.c:\\
  calculate scalar functions of lubrication matrix defined by 
  \begin{equation}
    \bm{L}_{\text{2B}}
    :=
    \bm{R}^{\text{exact}}_{\text{2B}}
    -
    \left(
      \bm{M}^{\infty}_{\text{2B}}
    \right)^{-1}
    .
  \end{equation}
  In this routine, the functions 
  {\tt twobody\_scalars\_res()} and 
  {\tt scalars\_minv\_[f,ft,fts]\_poly()} are called.
  The former is called twice to obtain scalar functions 
  for $12$- and $21$-pairs in $\bm{R}^{\text{exact}}_{\text{2B}}$, 
  and the latter is once for $(\bm{M}^{\infty}_{\text{2B}})^{-1}$.
  The scalar functions are in the dimensional form.
\item {\tt twobody\_scalars\_res()} in twobody.c:\\
  This routine handles whether lubrication form is used 
  for the exact solution (to improve the convergence), and 
  the scaling of the result scalar functions.
\end{itemize}


\chapter{Periodic Boundary Condition}
\label{chp:ewald}



\chapter{Brownian Dynamics}



\chapter{Connections and Bond Interactions}
\label{chp:bonds}

% bonds-groups : RYUON Reference Manual
% Copyright (C) 2008 Kengo Ichiki <kichiki@users.sourceforge.net>
% $Id: ref-bonds-groups.tex,v 1.1 2008/07/11 04:16:57 kichiki Exp $

\section{General Configurations}
\label{sec:ref-bonds-groups-general-configurations}
This section describes the code {\tt bonds-groups.c} 
handling bonds and connector vectors. 
The following is an excerption from the header file; 
{\footnotesize
\begin{verbatim}
struct BONDS {
  int n; // number of bonds

  <-- snip -->

  int *ia;
  // ia[n] : particle index of one end of the bond
  int *ib;
  // ib[n] : particle index of the other end of the bond
};

struct BONDS_GROUP {
  int np; // number of particles in this group
  int *ip;
  // ip[np] : particle index of the group particles
  int *bonds;
  // bonds[np-1] : independent bond list
};

struct BONDS_GROUPS {
  int n;                      // number of groups
  struct BONDS_GROUP **group; // grp[ng]
};
\end{verbatim}
}
(NOTE: this is still in development 
and not imported into {\tt libstokes} yet.)



\subsection{Backgroud}
Here, we discuss the generalization of the chain structure 
beyond the simple straight chains. 
For this purpose, we introduce particle indices 
$(\alpha_{i}, \beta_{i})$ for connection (bond) $i$. 
Because the connection is completely characterized 
by the link between the two particles, 
the list of particle-pairs $(\alpha_{i}, \beta_{i})$ 
is enough for arbitrary structure. 

Consider a system with $N$ particles. 
The positions of particles are given by 
$\{\bm{x}_{\mu}\}$ for $\mu = 0, \cdots, N-1$, 
where $N$ is the total number of particles in the system. 
Let $N_{b}$ be the number of bonds (connections) in the system. 
Each bond is characterized by the pair of particles $(\alpha_{i}, \beta_{i})$, 
where the bond index $i$ runs from $0$ to $N_{b}-1$. 
We also define the ``group'' 
which is a set containing particles connecting each other. 
Particles without connection belong to different groups. 
We denote the number of groups in the system by $N_{g}$. 


Contrast to the simple straight chain configuration, 
one extra care needs to be taken for general configurations. 
That is, for a single group of particles which are all connecting each other, 
sometimes the number of bonds (connections) becomes larger than 
the degrees of freedom of the group except for the center-of-mass, $N-1$. 
For straight chain configurations, the number of bonds are always $N-1$. 
This fact means that the extra bonds are redundant and 
the connector vectors for the redundant bonds can be 
derived from those for the independent bonds. 
See an example in Eq. (\ref{eq:ref-bonds-groups-redundant-connector-for-loop}) later.

The crucial point is forming the list of independent bonds. 
We denote the list by ${\tt bonds}[N-1]$. 
Detailed discussions for the construction of the list 
will be given in \S \ref{sec:ref-bonds-groups-independent-bond-list-algorithm}. 



Let us define the connector vector $\bm{Q}_{i}$ for the bond $i$ as 
\begin{equation}
  \bm{Q}_{i}
  =
  \bm{x}_{\alpha_{i}} - \bm{x}_{\beta_{i}}
  .
\end{equation}
Furthermore, we define the generalized (or canonical?) connector vector 
$\overline{\bm{Q}}_{i}$ where the index $i$ runs from $0$ to $N-1$. 
For a system with single group with $N$ particles, 
\begin{equation}
  \overline{\bm{Q}}_{i}
  =
  \bm{x}_{\alpha_{{\tt bonds}[i]}}
  -
  \bm{x}_{\beta_{{\tt bonds}[i]}}
  ,
\end{equation}
for $i = 0, \cdots, N-2$, and 
\begin{equation}
  \overline{\bm{Q}}_{N-1}
  =
  \bm{x}_{\text{COM}}
  :=
  \frac{1}{N}
  \sum_{\mu=0}^{N-1}
  \bm{x}_{\mu}
  ,
\end{equation}
where ${\tt bonds}[i]$ is the $i$th bond index 
of the independent bonds. 
Note that the number of elements of all connectors 
$\{\overline{\bm{Q}}_{i}\}$ 
is equal to that of all position vectors $\{\bm{x}_{\mu}\}$ 
and they have one-to-one correspondence. 
By introducing the transform matrix by $\overline{\mathsf{B}}$ 
whose dimension is $N\times N$, we can write 
\begin{equation}
  \overline{\bm{Q}}_{i}
  =
  \sum_{\mu=0}^{N-1}
  \overline{\mathsf{B}}_{i,\mu}
  \bm{x}_{\mu}
  .
  \label{eq:ref-bonds-groups-conversion-particle-wise}
\end{equation}
The details about the transformation will be discussed later 
in \S \ref{sec:ref-bonds-groups-conversion-pos-conn}. 



\subsection{Examples}
Before proceeding, we see several examples.

\subsubsection{Straight Configuration}
\begin{figure}
  \centering
  \includegraphics[width=7cm]{figures/ref-bonds-groups-straight2}
  \caption{
    Two straight chains with $N=7$.
  }
  \label{fig:ref-bonds-groups-straight2}
\end{figure}
\begin{table}[!htb]
  \begin{tabular}{ccc}
    bond index $i$ & $\alpha_{i}$ & $\beta_{i}$ \\
    \hline
    0 & 0 & 1 \\
    1 & 1 & 2 \\
    2 & 3 & 4 \\
    3 & 4 & 5 \\
    4 & 5 & 6
  \end{tabular}
  \caption{%
    Connection table for two straight chains 
    in Fig. \ref{fig:ref-bonds-groups-straight2}.
  }
  \label{tab:ref-bonds-groups-straight2}
\end{table}
Figure \ref{fig:ref-bonds-groups-straight2} shows an example of the system 
with $N=7$ particles forming two straight chains, that is, 
the number of groups is $N_{g} = 2$ and 
the number of bonds is $N_{b} = 5$. 
The connection table is given by Table \ref{tab:ref-bonds-groups-straight2}. 

\subsubsection{Star Configuration}
\begin{figure}
  \centering
  \includegraphics[width=5cm]{figures/ref-bonds-groups-star}
  \caption{
    A star configuration with $N=5$.
  }
  \label{fig:ref-bonds-groups-star}
\end{figure}
\begin{table}[!htb]
  \begin{tabular}{ccc}
    bond index $i$ & $\alpha_{i}$ & $\beta_{i}$ \\
    \hline
    0 & 0 & 1 \\
    1 & 0 & 2 \\
    2 & 0 & 3 \\
    3 & 0 & 4
  \end{tabular}
  \caption{%
    Connection table for the star configuration 
    in Fig. \ref{fig:ref-bonds-groups-star}.
  }
  \label{tab:ref-bonds-groups-star}
\end{table}
Figure \ref{fig:ref-bonds-groups-star} shows another example 
forming a star configuration with $N=5$ and $N_{b} = 4$, 
that is, all the bonds are independent. 
The connection table is given by Table \ref{tab:ref-bonds-groups-star}. 



\subsubsection{Loop Configuration}
\begin{figure}
  \centering
  \includegraphics[width=5cm]{figures/ref-bonds-groups-loop}
  \caption{
    A loop configuration with $N=6$.
  }
  \label{fig:ref-bonds-groups-loop}
\end{figure}
\begin{table}[!htb]
  \begin{tabular}{ccc|cccc}
    \multicolumn{3}{c|}{before} &
    \multicolumn{4}{c}{after sorting} \\
    $i$ & $\alpha_{i}$ & $\beta_{i}$ & 
    $i$ & $\alpha_{i}$ & $\beta_{i}$ & \\
    \hline
    0 & 0 & 1 & 0 & 0 & 1 & \\
    1 & 1 & 2 & 1 & 0 & 5 & redundant\\
    2 & 2 & 3 & 2 & 1 & 2 & \\
    3 & 3 & 4 & 3 & 2 & 3 & \\
    4 & 4 & 5 & 4 & 3 & 4 & \\
    5 & 5 & 0 & 5 & 4 & 5 & 
  \end{tabular}
  \caption{%
    Connection table for the star configuration 
    in Fig. \ref{fig:ref-bonds-groups-loop}.
    The Left-half is the table given by Fig. \ref{fig:ref-bonds-groups-loop} 
    and the right-half is after sorting by particle indices. 
  }
  \label{tab:ref-bonds-groups-loop}
\end{table}
Figure \ref{fig:ref-bonds-groups-loop} shows a loop configuration with $N=6$ particles. 
It is seen that the number of bonds $N_{b}$ is equal to $N$. 
This means that one of the bonds is redundant, that is, 
it can be derived from the other connector vectors. 
It is obvious that the connector vector for the 5th bond 
in Fig. \ref{fig:ref-bonds-groups-loop}, for example, can be obtained from the rest as 
\begin{equation}
  \bm{Q}_{5}
  :=
  \bm{x}_{5} - \bm{x}_{0}
  =
  -\sum_{i=0}^{4}\left(\bm{x}_{i+1} - \bm{x}_{i}\right)
  =
  -\sum_{i=0}^{4}\bm{Q}_{i}
  .
  \label{eq:ref-bonds-groups-redundant-connector-for-loop}
\end{equation}
Therefore, the number of independent connector vectors is $N-1$, as expected.

After sorting the connection table by the particle indices, 
we have the table shown in the right-half in Table \ref{tab:ref-bonds-groups-loop}. 
We drop the redundant bond $i=1$ and form the independent 
bond list {\tt bonds[]} as 
\begin{equation}
  {\tt bonds}[N-1] = \{0, 2, 3, 4, 5\}.
\end{equation}


\begin{figure}
  \centering
  \includegraphics[width=5cm]{figures/ref-bonds-groups-loop-connected}
  \caption{
    A loop configuration with extra connections from particle 0.
  }
  \label{fig:ref-bonds-groups-loop-connected}
\end{figure}
\begin{table}[!htb]
  \begin{tabular}{ccc|cccc}
    \multicolumn{3}{c|}{before} &
    \multicolumn{4}{c}{after sorting} \\
    $i$ & $\alpha_{i}$ & $\beta_{i}$ & 
    $i$ & $\alpha_{i}$ & $\beta_{i}$ & \\
    \hline
    0 & 0 & 1 & 0 & 0 & 1 & \\
    1 & 1 & 2 & 1 & 0 & 2 & redundant\\
    2 & 2 & 3 & 2 & 0 & 3 & redundant\\
    3 & 3 & 4 & 3 & 0 & 4 & redundant\\
    4 & 4 & 5 & 4 & 0 & 5 & redundant\\
    5 & 5 & 0 & 5 & 1 & 2 & \\
    6 & 0 & 2 & 6 & 2 & 3 & \\
    7 & 0 & 3 & 7 & 3 & 4 & \\
    8 & 0 & 4 & 8 & 4 & 5 & 
  \end{tabular}
  \caption{%
    Connection table for the star configuration
    in Fig. \ref{fig:ref-bonds-groups-loop-connected}.
    The Left-half is the table given by 
    Fig. \ref{fig:ref-bonds-groups-loop-connected} 
    and the right-half is after sorting by particle indices. 
  }
  \label{tab:ref-bonds-groups-loop-connected}
\end{table}
Figure \ref{fig:ref-bonds-groups-loop-connected} shows a loop configuration 
with extra connections with particle 0 to all the others. 
In this case, the number of bonds $N_{b}$ exceeds $N-1$ by 4 
(because we add three more connections into the looped configuration 
in Fig. \ref{fig:ref-bonds-groups-loop} which already has one redundant bond). 

After sorting the connection table by the particle indices, 
we have the table shown in the right-half in Table \ref{tab:ref-bonds-groups-loop-connected}. 
We drop the redundant bonds $i=1, 2, 3, 4$ and form the independent 
bond list {\tt bonds[]} as 
\begin{equation}
  {\tt bonds}[N-1] = \{0, 5, 6, 7, 8\}.
\end{equation}



\begin{figure}
  \centering
  \includegraphics[width=5cm]{figures/ref-bonds-groups-loop-connected2}
  \caption{
    A loop configuration with extra connections skipping one particle.
  }
  \label{fig:ref-bonds-groups-loop-connected2}
\end{figure}
\begin{table}[!htb]
  \begin{tabular}{ccc|cccc}
    \multicolumn{3}{c|}{before} &
    \multicolumn{4}{c}{after sorting} \\
    $i$ & $\alpha_{i}$ & $\beta_{i}$ & 
    $i$ & $\alpha_{i}$ & $\beta_{i}$ & \\
    \hline
    0  & 0 & 1 & 0  & 0 & 1 & \\
    1  & 1 & 2 & 1  & 0 & 2 & redundant\\
    2  & 2 & 3 & 2  & 0 & 4 & redundant\\
    3  & 3 & 4 & 3  & 0 & 5 & redundant\\
    4  & 4 & 5 & 4  & 1 & 2 & \\
    5  & 5 & 0 & 5  & 1 & 3 & redundant\\
    6  & 0 & 2 & 6  & 1 & 5 & redundant\\
    7  & 1 & 3 & 7  & 2 & 3 & \\
    8  & 2 & 4 & 8  & 2 & 4 & redundant\\
    9  & 3 & 5 & 9  & 3 & 4 & \\
    10 & 4 & 0 & 10 & 3 & 5 & redundant\\
    11 & 5 & 1 & 11 & 4 & 5 & 
  \end{tabular}
  \caption{%
    Connection table for the star configuration 
    in Fig. \ref{fig:ref-bonds-groups-loop-connected2}.
    The Left-half is the table given by 
    Fig. \ref{fig:ref-bonds-groups-loop-connected2} 
    and the right-half is after sorting by particle indices. 
  }
  \label{tab:ref-bonds-groups-loop-connected2}
\end{table}
Figure \ref{fig:ref-bonds-groups-loop-connected2} 
shows another loop configuration 
with 6 extra connections skipping one particle. 
In this case, the number of bonds $N_{b}$ exceeds $N-1$ by 7. 

After sorting the connection table by the particle indices, 
we have the table shown in the right-half in 
Table \ref{tab:ref-bonds-groups-loop-connected2}. 
We drop the redundant bonds shown in 
Table \ref{tab:ref-bonds-groups-loop-connected2}, 
bond list {\tt bonds[]} as 
\begin{equation}
  {\tt bonds}[N-1] = \{0, 4, 7, 9, 11\}.
\end{equation}


\begin{figure}
  \centering
  \includegraphics[width=8cm]{figures/ref-bonds-groups-loop-connected2-straight}
  \caption{
    A loop configuration with extra connections skipping one particle 
    with $N$ particles in a straight form. 
  }
  \label{fig:ref-bonds-groups-loop-connected2-straight}
\end{figure}
For general $N$, the the independent bond list 
in this type of configuration becomes 
\begin{equation}
  {\tt bonds}[N-1] = \{0, 4, 7, 9, \cdots, 2i+3, \cdots, 2(N-2)+3\},
\end{equation}
where $2i+3$ is the $i$th element (counting from 0) 
and $2(N-2)+3$ is the last ($N-2$)-th element. 
The difference between the 0th and first elements is 4, 
and that between the first and second is 3. 
Beyond this point, the differences are 2. 
To see this, let us see the configuration modified in a straight form 
in Fig. \ref{fig:ref-bonds-groups-loop-connected2-straight}. 

For the particle 0, there are 4 bonds and three of them are redundant. 
Therefore, in the sorted form, the second independent bond 
is the forth from the first independent bond. 
For the next particle 1, one of the 4 bonds is connecting to the particle 0 
which is already taken into account in the sorted form. 
Among the rest three, two are redundant, so that the third independent bond 
is the third from the second. 
For the next particle 2, two of the 4 bonds are connecting 
either particle 0 or 1. Only one of the rest two bonds is redundant 
and the fourth independent bond is the second  from the third. 
For the particles later, the situation is the same for 
the particle 2, so that the difference between the independent bonds are 2. 


\subsection{Algorithm}
\label{sec:ref-bonds-groups-independent-bond-list-algorithm}
Let us see the algorithm forming the independent bond list {\tt bonds}[] 
out of the set of particle indices $(\alpha_{i}, \beta_{i})$. 
This is implemented in the routine {\tt BONDS\_GROUPS\_make(b, np)}, 
where {\tt b} is a pointer of the struct {\tt BONDS} 
and {\tt np} is $N$, the number of particles in the system. 
The routine returns the struct {\tt BONDS\_GROUPS}. 


At the very beginning, we sort the list of particle indices 
$(\alpha_{i}, \beta_{i})$ as 
\begin{equation}
  \alpha_{i} < \beta_{i},
  \quad\text{ and }\quad
  \alpha_{i} < \alpha_{i+1}.
  \label{eq:ref-bonds-groups-BONDS-sort-alpha-beta}
\end{equation}
This is done by the routine {\tt BONDS\_sort\_by\_ia(b)}.

The first task is to decompose the particles into each group. 
This is done by the following three steps:
\begin{enumerate}
\item count the number of bonds for each particle (stored in {\tt nb[np]}),
\item obtain the bond indices for each particle (stored in {\tt bond[np][nb]}),
\item give independent group id for each particle (stored in {\tt gid[np]}).
\end{enumerate}
In {\tt gid[np]}, 0 is for the particles which is independent 
and has no bond (that is, ${\tt nb}[i]$ is 0 for the particle $i$). 

The third step is the crucial one, where the redundant bonds are also detected. 
Let us focus on the step. The core part of this is implemented in 
the routine {\tt BONDS\_GROUPS\_gid\_check()}. 
This routine is looking for the particles connected to 
the particle ``{\tt i}'' whose group id is ``{\tt ig}.''

First, a bond on the particle {\tt i} is picked. 
The bond index is {\tt ib}. 
Either $\alpha_{\tt ib}$ or $\alpha_{\tt ib}$ is equal to {\tt i} 
and the other particle (connecting to {\tt i}) is taken as {\tt k}. 
The group id {\tt gid[k]} would be either one of the three cases:
\begin{enumerate}
\item {\tt gid[k] == -1},
  where the particle {\tt k} is not assigned yet,
\item {\tt gid[k] == ig},
  where the particle {\tt k} is already assigned to the same group {\tt ig}, 
\item otherwise,
  the particle {\tt k} belongs to another group ($\neq{\tt ig}$).
\end{enumerate}
For the first case, we just add the particle {\tt k} to the group {\tt ig}, 
that is, define as {\tt gid[k] = ig}. We also need to check 
the connection from the particle {\tt k} by calling 
{\tt BONDS\_GROUPS\_gid\_check()} recursively. 

For the third case, we need to merge the two groups into one. 
In the present code, we change the latter group belonging to {\tt k} 
to the group {\tt ig} belonging to the particle {\tt i}. 
This is done by the routie {\tt BONDS\_GROUPS\_gid\_flip}. 
In this case, we don't need to check the connection for {\tt k} further, 
because {\tt k} has been checked (and assigned to some group). 

The second case is the important one. 
This means that the bond between the particles {\tt i} and {\tt k} 
forms a loop inside the same group {\tt ig}. 
We identify the bond as redundant. 
The redundant bonds are stored in the struct {\tt loop\_bonds} 
(which is defined only for the internal use). 
In this case, also, we don't need to check the connection. 


Therefore, after the third step to define {\tt gid[np]}, 
we also have the redundant bond list in the struct {\tt loop\_bonds *lb}. 
With these results, we are able to construct 
the struct {\tt BONDS\_GROUP} for each group 
which has the following elements
\begin{itemize}
\item {\tt np}, the number of particles in the group,
\item {\tt ip[np]}, the particle indices of the group particles,
\item {\tt bonds[np-1]}, the independent bond list for the group.
\end{itemize}
Note that, for the isolated particles ({\tt np = 1}), 
the independent bond list {\tt bonds} is set by NULL. 
The struct {\tt BONDS\_GROUPS} is a list containing 
numbers of struct {\tt BONDS\_GROUP} in the system. 

This algorithm forming the independent bond list {\tt bonds} 
has recursive call and the cost would be $O(N\ln N)$. 
But in the simulation, this is called only once at the beginning, 
so that it doesn't matter. 


\subsection{Conversions between Positions and Connectors}
\label{sec:ref-bonds-groups-conversion-pos-conn}
In terms of the struct {\tt BONDS\_GROUP} 
which has $({\tt np}, {\tt ip[np]}, {\tt bonds[np-1]})$, 
we can construct the conversions 
$\{\bm{x}_{\mu}\}\mapsto\{\overline{\bm{Q}}_{i}\}$ 
and $\{\overline{\bm{Q}}_{i}\}\mapsto\{\bm{x}_{\mu}\}$. 

First, we see the system with a single group. 
Because we only take the independent bonds, 
the bond index $i$ runs from 0 to $N-2$ 
(where the number of independent bonds is $N-1$), 
while the particle index $\mu$ runs from 0 to $N-1$ 
(where the number of particles is $N$). 
Adding the $(N-1)$-th element of the connector vector 
$\{\overline{\bm{Q}}_{i}\}$ by the center of mass $\bm{x}_{\text{COM}}$ as 
\begin{equation}
  \overline{\bm{Q}}_{N-1}
  :=
  \bm{x}_{\text{COM}}
  =
  \frac{1}{N}
  \sum_{i=0}^{N-1}
  \bm{x}_{{\tt ip}[i]}
  ,
  \label{eq:ref-bonds-groups-last-connector-vector-COM}
\end{equation}
(therefore, precisely speaking, $\bm{x}_{\text{COM}}$ is not 
the center of \textit{mass} though), 
the numbers of elements of both $\{\bm{x}_{\mu}\}$ 
and $\{\overline{\bm{Q}}_{i}\}$ become the same. 

The transformations between them can be constructed as follows. 
By definition of connector vectors, 
\begin{equation}
  \overline{\bm{Q}}_{i}
  =
  \bm{x}_{\alpha_{{\tt bonds}[i]}} - \bm{x}_{\beta_{{\tt bonds}[i]}},
\end{equation}
for $i = 0, \cdots, N-2$. 
The last element of the connector vector $\overline{\bm{Q}}_{N-1}$ is given by 
Eq. (\ref{eq:ref-bonds-groups-last-connector-vector-COM}). 
This is implemented in the routine {\tt BONDS\_pos\_to\_conn\_1}. 
Note that, for the isolated particle, 
we define the (generalized) connector vector by 
the position itself as 
\begin{equation}
  \overline{\bm{Q}}_{0} = \bm{x}_{0}.
\end{equation}
As we see, the relation is linear and the transformation can be 
expressed by the matrix as 
\begin{equation}
  \overline{\bm{Q}}
  =
  \mathscr{B}\cdot\bm{x},
\end{equation}
where $\overline{\bm{Q}}$ and $\bm{x}$ are the generalized vector 
with $3N$ dimension for $\{\overline{\bm{Q}}_{i}\}$ and $\{\bm{x}_{\mu}\}$, 
respectively, and $\mathscr{B}$ is 
$3N\times 3N$ matrix based on the $N\times N$ matrix $\overline{\mathsf{B}}$ 
in Eq. (\ref{eq:ref-bonds-groups-conversion-particle-wise}). 


The opposite transformation 
from $\overline{\bm{Q}}_{i}$ to $\bm{x}_{\mu}$ can be obtained 
from the inversion matrix $\mathscr{B}^{-1}$. 
Alternatively (and directly), 
we can calculate the position vectors from the connector vectors 
(and the center-of-mass vector) as follows:
For the isolated particle, 
\begin{equation}
  \bm{x}_{0} = \overline{\bm{Q}}_{0}.
\end{equation}
For the group with $N(> 1)$ particles, 
we take the two steps: 
\begin{enumerate}
\item at the first step, $\bm{x}_{\mu}$ is formed
  with respect to the origin at the $\bm{x}_{\alpha_{{\tt bonds}[0]}}$,
\item then, adjust the origin 
  with respect to $\bm{x}_{\text{COM}} = \overline{\bm{Q}}_{N-1}$. 
\end{enumerate}

At the first step, 
therefore, we define 
\begin{equation}
  \bm{x}_{\alpha_{{\tt bonds}[0]}} = \bm{0},
\end{equation}
as an initial condition. 
After it, for the independent bond $i$ from 0 to $N-2$, 
\begin{equation}
  \bm{x}_{\beta_{{\tt bonds}[i]}}
  =
  \bm{x}_{\alpha_{{\tt bonds}[i]}}
  -
  \overline{\bm{Q}}_{i}
  .
\end{equation}
To make this procedure work, 
we should assure that 
$\bm{x}_{\alpha_{{\tt bonds}[i]}}$ has been defined 
by the $(i-1)$-th step. 
This condition is satisfied from the fact that 
\begin{itemize}
\item {\tt bonds[]} don't have redundant bonds,
\item the number of the independent bond is {\tt np-1}, 
\item the particle indices $\alpha_{i}$ and $\beta_{i}$ are 
  sorted as in Eq. (\ref{eq:ref-bonds-groups-BONDS-sort-alpha-beta}),
\item {\tt bonds[]} itself is sorted as 
  ${\tt bonds}[i] < {\tt bonds}[i+1]$.
\end{itemize}

After the first step, 
we calculate $\widetilde{\bm{x}}_{\text{COM}}$ as 
\begin{equation}
  \widetilde{\bm{x}}_{\text{COM}}
  =
  \frac{1}{N}
  \sum_{i=0}^{N-1}
  \bm{x}_{{\tt ip}[i]}
  \ 
  \left(
    =
    \frac{1}{N}
    \sum_{i=0}^{N-2}
    \bm{x}_{\beta_{{\tt bonds}[i]}}
  \right)
  ,
\end{equation}
and adjust the center of mass to $\bm{x}_{\text{COM}}$ as 
\begin{equation}
  \bm{x}_{{\tt ip}[i]}
  =
  \bm{x}_{{\tt ip}[i]}
  -
  \widetilde{\bm{x}}_{\text{COM}}
  +
  \bm{x}_{\text{COM}}
  ,
\end{equation}
for $i = 0, \cdots, N-1$.

It should be noted that 
the calculation of this rather tedeous procedure is $O(N)$, 
while the inversion $\mathsf{A}^{-1}$ usually costs $O(N^3)$. 

For the system containing many groups at the same time, 
we can apply the above procedure for each group separately, 
and the overall calculation is still $O(N)$. 
This system-wide transformations are implimented in the routines 
{\tt BONDS\_pos\_to\_conn} and {\tt BONDS\_conn\_to\_pos}. 






%\bibliographystyle{abbrv}
%\bibliographystyle{alpha}
%\bibliographystyle{apalike}
%\bibliographystyle{ieeetr}
%\bibliographystyle{plain}
\bibliographystyle{siam}
%\bibliographystyle{unsrt}
%\bibliographystyle{prsty}
\bibliography{mine,stokes,fmm,opensource}

\end{document}
